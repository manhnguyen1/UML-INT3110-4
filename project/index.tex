\documentclass{article}
\usepackage[utf8]{vietnam}
\usepackage{amsthm}
\usepackage{amsmath}
\usepackage{amsfonts}
\usepackage{amssymb}
\usepackage{graphicx}
\usepackage{url}
\usepackage{cases}

\title{\textbf{Phân tích thiết kế hệ thống "Đăng ký môn học"}}
\author{
  Ngô Quang Dương
}
\date{\today}

\begin{document}

\maketitle

\begin{abstract}
\end{abstract}

\section{Mở đầu}

  \subsection{Đặt vấn đề}

  \subsection{Hệ thống hiện tại}

  \subsection{Hướng giải quyết}

\section{Thu thập và phân tích yêu cầu}
  
  \subsection{Bảng thuật ngữ}
    \begin{itemize}
      \item \textbf{Người dùng}: Những người có tài khoản trong hệ thống đăng ký môn học.
      \item \textbf{Chuyên viên}: Những người làm việc ở phòng công tác sinh viên.
      \item \textbf{Lớp môn học}: Một môn học có thể được chia ra làm nhiều lớp. Chẳng hạn với môn cơ sở dữ liệu (mã môn học là \textbf{INT2207}) có các lớp \textbf{INT2207 1}, \textbf{INT2207 2}, \textbf{INT2207 3}, \ldots
      \item \textbf{Buổi lý thuyết}: Mọi lớp học đều có duy nhất một buổi lý thuyết.
      \item \textbf{Buổi thực hành}: Một lớp học có thể có nhiều hoặc không có buổi thực hành nào.
    \end{itemize}
  
  \subsection{Tác nhân hệ thống}
    \begin{itemize}
      \item Quản trị hệ thống.
      \item Sinh viên.
      \item Chuyên viên.
      \item Giảng viên.
    \end{itemize}
  
  \subsection{Yêu cầu chức năng}
    \paragraph{Chức năng chung:}
    \begin{itemize}
      \item Đăng nhập/đăng xuất.
      \item Chỉnh sửa thông tin tài khoản.
    \end{itemize}

    \paragraph{Chức năng dành cho quản trị hệ thống:}
    \begin{itemize}
      \item Quản lý người dùng.
      \begin{itemize}
        \item Tìm kiếm người dùng.
        \item Tạo người dùng mới.
        \item Chỉnh sửa thông tin.
        \item Xóa người dùng.
      \end{itemize}
      \item Quản lý môn học:
      \begin{itemize}
        \item Tìm kiếm môn học.
        \item Tạo môn học/lớp môn học mới.
        \item Chỉnh sửa thông tin môn học/lớp môn học.
        \item Xóa môn học/lớp môn học.
      \end{itemize}
      \item Quản lý lớp học:
      \begin{itemize}
        \item Tìm kiếm lớp học.
        \item Tạo lớp học mới.
        \item Chỉnh sửa thông tin lớp học.
        \item Xóa lớp học.
      \end{itemize}
      \item Mở/đóng hệ thống:
      \begin{itemize}
        \item Cho sinh viên đăng ký môn học.
        \item Cho giảng viên sắp xếp thời khóa biểu.
      \end{itemize}
    \end{itemize}

    \paragraph{Chức năng dành cho sinh viên và chuyên viên:}
    \begin{itemize}
      \item Tìm kiếm môn học.
      \item Đăng ký môn học.
      \begin{itemize}
        \item Đăng ký môn học mới.
        \item Bỏ môn học đã chọn.
        \item Xem danh sách các môn đã đăng ký.
      \end{itemize}
    \end{itemize}

    \paragraph{Chức năng dành cho giảng viên:}
    \begin{itemize}
      \item Tìm kiếm lớp học.
      \item Chọn lớp giảng dạy.
      \item Chọn thời khóa biểu.
      \item Chọn phòng học.
      \item Xem danh sách các lớp đã nhận.
    \end{itemize}

  \subsection{Yêu cầu phi chức năng}
    \paragraph{
      \textnormal{
        Qua khảo sát đối với người dùng là sinh viên, hệ thống cần được đáp ứng các yêu cầu sau:
      }
    }
    \begin{itemize}
      \item Kết nối nhanh.
      \item Thời gian thực.
      \item Giao diện dễ sử dụng.
      \item Dễ tìm kiếm môn học cần đăng ký.
    \end{itemize}
  
  \subsection{Điều kiện ràng buộc}
    \paragraph{Đối với sinh viên và chuyên viên:}
    \begin{itemize}
      \item Không đăng ký quá 2 môn giáo dục thể chất.
      \item Không đăng ký môn học đã qua với điểm cao hơn $ D $.
      \item Không đăng ký 2 môn học trùng thời khóa biểu.
      \item Số tín chỉ không vượt quá $ 40 $.
    \end{itemize}

    \paragraph{Đối với giảng viên:}
    \begin{itemize}
      \item Không nhận hai lớp bị trùng thời khóa biểu.
    \end{itemize}

\section{Đặc tả yêu cầu}

  \subsection{Sơ đồ use case}

  \begin{figure}[!ht]
    \centering
    \includegraphics[scale=0.4]{../pictures/projectdiagrams/uc.jpg}
    \caption{Sơ đồ use case}
  \end{figure}

  \begin{figure}[!ht]
    \centering
    \includegraphics[scale=0.4]{../pictures/projectdiagrams/uc-destructing.jpg}
    \caption{Sơ đồ use case phân rã}
  \end{figure}

  \subsection{Đặc tả use case}

  \subsection{Sơ đồ hoạt động}

\section{Phân tích tĩnh}

  \subsection{Xác định lớp}

  \subsection{Quan hệ giữa các lớp/sơ đồ lớp}

  \subsection{Lớp phân tích}

  \subsection{Xác định thuộc tính}

  \subsection{Xác định phương thức}

\section{}

\end{document}
